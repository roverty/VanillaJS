%%%%%%%%%%%%%%%%%%%%%%%%%%%%%%%%%%%%%%%%%%%%%%%%%%%%%%%%%%%%%%%%%%%%
%% I, the copyright holder of this work, release this work into the
%% public domain. This applies worldwide. In some countries this may
%% not be legally possible; if so: I grant anyone the right to use
%% this work for any purpose, without any conditions, unless such
%% conditions are required by law.
%%%%%%%%%%%%%%%%%%%%%%%%%%%%%%%%%%%%%%%%%%%%%%%%%%%%%%%%%%%%%%%%%%%%

\documentclass[aspectratio=169]{beamer}
\usetheme[faculty=fi]{fibeamer}
\usepackage[utf8]{inputenc}
\usepackage[
  main=spanish, %% By using `czech` or `slovak` as the main locale
                %% instead of `english`, you can typeset the
                %% presentation in either Czech or Slovak,
                %% respectively.
  english,czech, slovak %% The additional keys allow foreign texts to be
]{babel}        %% typeset as follows:
%%
%%   \begin{otherlanguage}{czech}   ... \end{otherlanguage}
%%   \begin{otherlanguage}{slovak}  ... \end{otherlanguage}
%%
%% These macros specify information about the presentation
\title{Curso JS \\ Itersemestral 2021-1 [Online]} %% that will be typeset on the
\subtitle{Tema 04: Programación funcional} %% title page.
\author{
  Rodrigo Francisco \\
}
%% These additional packages are used within the document:
\usepackage{ragged2e}  % `\justifying` text
\usepackage{booktabs}  % Tables
\usepackage{tabularx}
\usepackage{tikz}      % Diagrams
\usetikzlibrary{calc, shapes, backgrounds}
\usepackage{amsmath, amssymb}
\usepackage{url}       % `\url`s
\usepackage{listings}  % Code listings
\usepackage{multicol}
\usepackage{float}
\usepackage{wrapfig}
\usepackage{subcaption}
\graphicspath{ {../introduccionJS.assets/} }
\frenchspacing

\begin{document}
  \shorthandoff{-}
  \frame[c]{\maketitle}

  \AtBeginSection[]{% Print an outline at the beginning of sections
    \begin{frame}<beamer>
      %\frametitle{Contenido de la sección \thesection}
      \frametitle{Agenda}
        {\small \tableofcontents[currentsection]}
    \end{frame}}

  \begin{darkframes}
    \section{Iterators}
    \begin{frame}{Lo que no es Programación funcional}
      \begin{itemize}
        \item Ciclos
        \begin{itemize}
          \item while
          \item do ... while
          \item for
          \item etc.
        \end{itemize}
        \item Declaración de variables con \textbf{var} o \textbf{let}
        \item Funciones vacías
        \item Mutación de objetos (Por ejemplo: \textbf{o.x = 5;})
        \item Métodos mutadores de objetos como
        \begin{itemize}
          \item pop
          \item push
          \item shift
          \item sort
          \item splice
        \end{itemize}
      \end{itemize}
    \end{frame}
    \section{Generators}
    \section{Composición de funciones}
   
  \end{darkframes}

\end{document}
